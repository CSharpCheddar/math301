\documentclass[12pt, letter]{article}

\usepackage{amsmath}
\usepackage{pgfplots}
\usepackage{slashbox}

\title{Homework 9}
\author{Martin Mueller}
\date{Due: April $8^{th}$, 2019}

\begin{document}
\maketitle

\begin{center}
	\underline{\textbf{Chapter 5}}
\end{center}

\textbf{1. (25 points) A fair coin is tossed four times. The random variable $X$ is the number of heads in the first three tosses and the random variable $Y$ is the number of heads in the last three tosses.}

\qquad \textbf{a) What is the joint pmf of $X$ and $Y$?}
\begin{center}
	First, let's take a look at all possible outcomes of flipping a coin 4 times:
	\newline
	\newline
	\begin{tabular}{c|c}
		Outcome & ($X$,$Y$) \\
		\hline
		TTTT & (0,0) \\
		TTTH & (0,1) \\
		TTHT & (1,1) \\
		TTHH & (1,2) \\
		THTT & (1,1) \\
		THTH & (1,2) \\
		THHT & (2,2) \\
		THHH & (2,3) \\
		HTTT & (1,0) \\
		HTTH & (1,1) \\
		HTHT & (1,2) \\
		HTHH & (2,2) \\
		HHTT & (2,1) \\
		HHTH & (2,2) \\
		HHHT & (3,2) \\
		HHHH & (3,3)
	\end{tabular}
	
	\pagebreak
	
	Now we can construct a pmf of $X$ and $Y$ in the form of a table by simply taking the number of times each ordered pair of $X$ and $Y$ pops up and dividing it by the total number of outcomes:
	\newline
	\newline
	\def\arraystretch{1.5}
	\begin{tabular}{c|cccc|c}
		\backslashbox{$Y$}{$X$} & 0 & 1 & 2 & 3 & \\
		\hline
		0 & $\frac{1}{16}$ & $\frac{1}{16}$ & 0 & 0 & $\frac{2}{16}$ \\
		1 & $\frac{1}{16}$ & $\frac{3}{16}$ & $\frac{1}{16}$ & 0 & $\frac{5}{16}$ \\
		2 & 0 & $\frac{3}{16}$ & $\frac{3}{16}$ & $\frac{1}{16}$ & $\frac{7}{16}$ \\
		3 & 0 & 0 & $\frac{1}{16}$ & $\frac{1}{16}$ & $\frac{2}{16}$ \\
		\hline
		& $\frac{2}{16}$ & $\frac{7}{16}$ & $\frac{5}{16}$ & $\frac{2}{16}$ & 1
	\end{tabular}
\end{center}

\qquad \textbf{b) What are the marginal pmf’s of $X$ and $Y$?}
\begin{center}
	The marginal pmf's of $X$ and $Y$ are listed in the bottom and right sides of the above table respectively. These represent the probabilities of a single variable taking a particular value regardless of the value of the other.
\end{center}

\qquad \textbf{c) Are the random variables $X$ and $Y$ independent? Explain.}
\begin{center}
	If the two variables are independent, then $P(X,Y) = P(X) \cdot P(Y)$ for all $X$ and $Y$. Let's start by testing $P(X = 0)$ and $P(Y = 0)$:
	$$P(X = 0) = \frac{2}{16}$$
	$$P(Y = 0) = \frac{2}{16}$$
	$$P(X = 0, Y = 0) = \frac{1}{16}$$
	$$P(X = 0) \cdot P(Y = 0) = \frac{2}{16} \cdot \frac{2}{16} = \frac{1}{64}$$
	As you can see, we have already disproven that $X$ and $Y$ are not independent for 1 case, therefore, they are not independent at all.
\end{center}

\pagebreak

\qquad \textbf{d) Find Cov($X$,$Y$).}
\begin{center}
	\begin{align*}
		\mu_{x} &= 0\left(\frac{2}{16}\right) + 1\left(\frac{7}{16}\right) + 2\left(\frac{5}{16}\right) + 3\left(\frac{2}{16}\right) = 1.4375 \\
		\mu_{y} &= 0\left(\frac{2}{16}\right) + 1\left(\frac{5}{16}\right) + 2\left(\frac{7}{16}\right) + 3\left(\frac{2}{16}\right) = 1.5625
	\end{align*}
	\begin{align*}
		Cov(X,Y) &= E(XY) - \mu_{x} \mu_{y} \\
		&= 0 \cdot 0 \cdot \left(\frac{1}{16}\right) + 0 \cdot 1 \cdot \left(\frac{1}{16}\right) + 0 \cdot 2 \cdot (0) + 0 \cdot 3 \cdot (0) \\
		&+ 1 \cdot 0 \cdot \left(\frac{1}{16}\right) + 1 \cdot 1 \cdot \left(\frac{3}{16}\right) + 1 \cdot 2 \cdot \left(\frac{1}{16}\right) + 1 \cdot 3 \cdot (0) \\
		&+ 2 \cdot 0 \cdot (0) + 2 \cdot 1 \cdot \left(\frac{3}{16}\right) + 2 \cdot 2 \cdot \left(\frac{3}{16}\right) + 2 \cdot 3 \cdot \left(\frac{1}{16}\right) \\
		&+ 3 \cdot 0 \cdot (0) + 3 \cdot 1 \cdot (0) + 3 \cdot 2 \cdot \left(\frac{1}{16}\right) + 3 \cdot 3 \cdot \left(\frac{1}{16}\right) \\
		&- (1.4375)(1.5625) \\
		&= \boxed{0.5039}
	\end{align*}
\end{center}

\qquad \textbf{e) Find the correlation coefficient of $X$ and $Y$.}
\begin{center}
	\begin{align*}
		\sigma_{x} &= \sqrt{E[X^{2}] - \mu_{x}^{2}} \\
		&= \sqrt{0^{2}\left(\frac{2}{16}\right) + 1^{2}\left(\frac{7}{16}\right) + 2^{2}\left(\frac{5}{16}\right) + 3^{2}\left(\frac{2}{16}\right) - (1.4375)^{2}} \\
		&= 0.8638 \\
		\sigma_{y} &= \sqrt{E[Y^{2}] - \mu_{y}^{2}} \\
		&= \sqrt{0^{2}\left(\frac{2}{16}\right) + 1^{2}\left(\frac{5}{16}\right) + 2^{2}\left(\frac{7}{16}\right) + 3^{2}\left(\frac{2}{16}\right) - (1.5625)^{2}} \\
		&= 0.8638
	\end{align*}
	$$\rho_{(x,y)} = \frac{Cov(X,Y)}{\sigma_{x}\sigma_{y}} = \frac{0.5039}{(0.8638)(0.8638)} = \boxed{0.6753}$$
\end{center}

\textbf{4. (30 points) Suppose $X$ and $Y$ are discrete random variables with joint pmf:}
\begin{center}
	\begin{align*}
		p(x,y) &= kxy & x=1,2,&3 \text{ and } y=2,3,4 \\
		&=0 & &\text{otherwise}
	\end{align*}
\end{center}

\qquad \textbf{a) Find the value of $k$.}
\begin{center}
	First, let's create a pmf for $x$ and $y$:
	\newline
	\newline
	\def\arraystretch{1.5}
	\begin{tabular}{c|ccc|c}
		\backslashbox{$y$}{$x$} & 1 & 2 & 3 & \\
		\hline
		2 & 2$k$ & 4$k$ & 6$k$ & 12$k$ \\
		3 & 3$k$ & 6$k$ & 9$k$ & 18$k$ \\
		4 & 4$k$ & 8$k$ & 12$k$ & 24$k$ \\
		\hline
		& 9$k$ & 18$k$ & 27$k$ & 1
	\end{tabular}
	\newline
	\newline
	Now we can take the marginal pmf of $x$ or $y$, (let's do $x$), and set the sum of the terms equal to 1:
	\begin{align*}
		12k + 18k + 24k &= 1 \\
		54k &= 1 \\
		k &= \boxed{\frac{1}{54}}
	\end{align*}
\end{center}

\qquad \textbf{b) Find $P(X < Y)$.}
\begin{center}
	Looking at our table, we can take the sum of all the cases where $x$ is less than $y$ and add the probabilities plugging in $\frac{1}{54}$ for $k$:
	$$\frac{2}{54} + \frac{3}{54} + \frac{6}{54} + \frac{4}{54} + \frac{8}{54} + \frac{12}{54} = \boxed{\frac{35}{54}}$$
\end{center}

\pagebreak

\qquad \textbf{c) Find the marginal pmf’s of $X$ and $Y$.}
\begin{center}
	Plugging in $\frac{1}{54}$ for $k$ in our above table, we can get the following:
	\newline
	\newline
	\def\arraystretch{1.5}
	\begin{tabular}{c|ccc|c}
		\backslashbox{$y$}{$x$} & 1 & 2 & 3 & \\
		\hline
		2 & $\frac{2}{54}$ & $\frac{4}{54}$ & $\frac{6}{54}$ & $\frac{12}{54}$ \\
		3 & $\frac{3}{54}$ & $\frac{6}{54}$ & $\frac{9}{54}$ & $\frac{18}{54}$ \\
		4 & $\frac{4}{54}$ & $\frac{8}{54}$ & $\frac{12}{54}$ & $\frac{24}{54}$ \\
		\hline
		& $\frac{9}{54}$ & $\frac{18}{54}$ & $\frac{27}{54}$ & 1
	\end{tabular}
\end{center}

\qquad \textbf{d) Find Cov($X$,$Y$).}
\begin{center}
	\begin{align*}
		\mu_{x} &= 1\left(\frac{9}{54}\right) + 2\left(\frac{18}{54}\right) + 3\left(\frac{27}{54}\right) = 2.3333 \\
		\mu_{y} &= 2\left(\frac{12}{54}\right) + 3\left(\frac{18}{54}\right) + 4\left(\frac{24}{54}\right) = 3.2222
	\end{align*}
	\begin{align*}
		Cov(X,Y) &= E(XY) - \mu_{x} \mu_{y} \\
		&= 1 \cdot 2 \cdot \left(\frac{2}{54}\right) + 1 \cdot 3 \cdot \left(\frac{3}{54}\right) + 1 \cdot 4 \cdot \left(\frac{4}{54}\right) \\
		&+ 2 \cdot 2 \cdot \left(\frac{4}{54}\right) + 2 \cdot 3 \cdot \left(\frac{6}{54}\right) + 2 \cdot 4 \cdot \left(\frac{8}{54}\right) \\
		&+ 3 \cdot 2 \cdot \left(\frac{6}{54}\right) + 3 \cdot 3 \cdot \left(\frac{9}{54}\right) + 3 \cdot 4 \cdot \left(\frac{12}{54}\right) \\
		&- (2.3333)(3.2222) \\
		&= \boxed{0.0001592}
	\end{align*}
\end{center}

\pagebreak

\qquad \textbf{e) Find the correlation coefficient of $X$ and $Y$.}
\begin{center}
	\begin{align*}
		\sigma_{x} &= \sqrt{E[x^{2}] - \mu_{x}^{2}} \\
		&= \sqrt{1^{2}\left(\frac{9}{54}\right) + 2^{2}\left(\frac{18}{54}\right) + 3^{2}\left(\frac{27}{54}\right) - (2.3333)^{2}} \\
		&= 0.7455 \\
		\sigma_{y} &= \sqrt{E[y^{2}] - \mu_{y}^{2}} \\
		&= \sqrt{2^{2}\left(\frac{12}{54}\right) + 3^{2}\left(\frac{18}{54}\right) + 4^{2}\left(\frac{24}{54}\right) - (3.2222)^{2}} \\
		&= 0.7858
	\end{align*}
	$$\rho_{(x,y)} = \frac{Cov(X,Y)}{\sigma_{x}\sigma_{y}} = \frac{0.0001592}{(0.7455)(0.7858)} = \boxed{0.0002718}$$
\end{center}

\qquad \textbf{f) Are $X$ and $Y$ independent? Explain.}
\begin{center}
	Let's check this by verifying that $P(x,y) = P(x) \cdot P(y)$ for all $x$ and $y$:
	\begin{align*}
		P(1,2) = \frac{2}{54} & & P(1) \cdot P(2) = \frac{2}{54} \\
		P(1,3) = \frac{3}{54} & & P(1) \cdot P(3) = \frac{3}{54} \\
		P(1,4) = \frac{4}{54} & & P(1) \cdot P(4) = \frac{4}{54} \\
		P(2,2) = \frac{4}{54} & & P(2) \cdot P(2) = \frac{4}{54} \\
		P(2,3) = \frac{6}{54} & & P(2) \cdot P(3) = \frac{6}{54} \\
		P(2,4) = \frac{8}{54} & & P(2) \cdot P(4) = \frac{8}{54} \\
		P(3,2) = \frac{6}{54} & & P(3) \cdot P(2) = \frac{6}{54} \\
		P(3,3) = \frac{9}{54} & & P(3) \cdot P(3) = \frac{9}{54} \\
		P(3,4) = \frac{12}{54} & & P(3) \cdot P(4) = \frac{12}{54} \\
	\end{align*}
	$\boxed{\text{Yes}}$, $X$ and $Y$ are independent.
\end{center}
\end{document}