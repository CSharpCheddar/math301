\documentclass[12pt, letter]{article}

\usepackage{amsmath}
\usepackage{slashbox}

\title{Homework 10}
\author{Martin Mueller}
\date{Due: April $19^{th}$, 2019}

\begin{document}
\maketitle

\begin{center}
	\underline{\textbf{Chapter 5}}
\end{center}

\textbf{7. (15 points) There are three traffic lights on my way to work. Let $X_{1}$ be the number of lights at which I must stop, and suppose the probability distribution of $X_{1}$ as follows:}
\begin{center}
	\def\arraystretch{1.2}
	\begin{tabular}{|c|c|c|c|c|}
		\hline
		$X_{1}$ & 0 & 1 & 2 & 3 \\
		\hline
		$p(X_{1})$ & .2 & .3 & .3 & .2 \\
		\hline
	\end{tabular}
\end{center}
\textbf{Let $X_{2}$ be the number of lights at which I must stop on the way home; $X_{2}$ is independent of $X_{1}$. Assume that $X_{2}$ has the same probability distribution as $X_{1}$, so that $X_{1}$ and $X_{2}$ is a random sample of size $n = 2$.}

\qquad \textbf{a) Let $T = X_{1} + X_{2}$. Determine the probability distribution of $T$.}
\begin{center}
	Since $X_{1}$ and $X_{2}$ are independent, we can just multiply the probabilities together to get the probability of $X_{1}$ and $X_{2}$ each taking on a particular value. Doing that, we can create the following table:
	\newline
	\newline
	\def\arraystretch{1.2}
	\begin{tabular}{c|cccc|c}
		\backslashbox{$X_{2}$}{$X_{1}$} & 0 & 1 & 2 & 3 & \\
		\hline
		0 & .04 & .06 & .06 & .04 & .20 \\
		1 & .06 & .09 & .09 & .06 & .30 \\
		2 & .06 & .09 & .09 & .06 & .30 \\
		3 & .04 & .06 & .06 & .04 & .20 \\
		\hline
		& .20 & .30 & .30 & .20 & 1
	\end{tabular}
	
	\pagebreak
	
	Using that table, we can determine the probability that the sum of $X_{1}$ and $X_{2}$ will be a particular number by adding the probabilities together. Doing this, we can create the following probability distribution for $T$:
	\newline
	\newline
	\def\arraystretch{1.2}
	\begin{tabular}{|c|c|c|c|c|c|c|c|}
		\hline
		$T$ & 0 & 1 & 2 & 3 & 4 & 5 & 6 \\
		\hline
		$p(T)$ & .04 & .12 & .21 & .26 & .21 & .12 & .04 \\
		\hline
	\end{tabular}
\end{center}

\qquad \textbf{b) Find $E(T)$. How does it relate to the population mean $\mu$?}
\begin{center}
	\begin{align*}
		E(T) &= 0(.04) + 1(.12) + 2(.21) + 3(.26) + 4(.21) + 5(.12) + 6(.04) \\
		&= \boxed{3}
	\end{align*}
	\begin{align*}
		E(T) &= n \mu \\
		\mu &= \frac{E(T)}{n} \\
		&= \frac{3}{2} \\
		&= \boxed{1.5}
	\end{align*}
\end{center}

\qquad \textbf{c) Find $V(T)$. How does it relate to the population variance $\sigma^{2}$?}
\begin{center}
	\begin{align*}
		V(T) &= 0^{2}(.04) + 1^{2}(.12) + 2^{2}(.21) + 3^{2}(.26) + 4^{2}(.21) + 5^{2}(.12) + 6^{2}(.04) \\
		&= \boxed{11.1}
	\end{align*}
	Since both $X_{1}$ and $X_{2}$ have the same distribution:
	\begin{align*}
		V(T) &= n \sigma^{2} \\
		\sigma^{2} &= \frac{V(T)}{n} \\
		&= \frac{11.1}{2} \\
		&= \boxed{5.55}
	\end{align*}
\end{center}

\textbf{8. (15 points) A random sample of size 2 is taken from a population with the following pmf:}
\begin{center}
	\def\arraystretch{1.2}
	\begin{tabular}{|c|c|c|c|c|}
		\hline
		$X$ & 0 & 1 & 2 & 3 \\
		\hline
		$p(X)$ & .1 & .2 & .4 & .3 \\
		\hline
	\end{tabular}
\end{center}

\qquad \textbf{a) Find the probability distribution of the sample mean $\bar{X}$.}
\begin{center}
	$$P(X = x) = p(x_{1})p(x_{2})$$
	\newline
	\newline
	\def\arraystretch{1.2}
	\begin{tabular}{c|cccc|c}
		\backslashbox{$X_{2}$}{$X_{1}$} & 0 & 1 & 2 & 3 & \\
		\hline
		0 & .01 & .02 & .04 & .03 & .10 \\
		1 & .02 & .04 & .08 & .06 & .20 \\
		2 & .04 & .08 & .16 & .12 & .40 \\
		3 & .03 & .06 & .12 & .09 & .30 \\
		\hline
		& .10 & .20 & .40 & .30 & 1
	\end{tabular}
	
	$$\bar{X} = \frac{X_{1} + X_{2}}{2}$$
	
	\def\arraystretch{1.2}
	\begin{tabular}{|c|c|c|c|c|c|c|c|}
		\hline
		$\bar{X}$ & 0.0 & 0.5 & 1.0 & 1.5 & 2.0 & 2.5 & 3.0 \\
		\hline
		$p(\bar{x})$ & .01 & .04 & .12 & .22 & .28 & .24 & .09 \\
		\hline
	\end{tabular}
\end{center}

\qquad \textbf{b) Find $E(\bar{X})$. How does it relate to the population mean?}
\begin{center}
	\begin{align*}
		E(\bar{X}) &= 0.0(.01) + 0.5(.04) + 1.0(.12) + 1.5(.22) \\
		& \qquad + 2.0(.28) + 2.5(.24) + 3.0(.09) \\
		&= \boxed{2.08}
	\end{align*}
	$$\boxed{E(\bar{X}) = \mu_{\bar{X}} = \mu}$$
\end{center}

\pagebreak

\qquad \textbf{c) Find $V(\bar{X})$. How does it relate to the population variance?}
\begin{center}
	\begin{align*}
		V(\bar{X}) &= 0.0^{2}(.01) + 0.5^{2}(.04) + 1.0^{2}(.12) + 1.5^{2}(.22) \\
		& \qquad + 2.0^{2}(.28) + 2.5^{2}(.24) + 3.0^{2}(.09) \\
		&= \boxed{4.055}
	\end{align*}
	\begin{align*}
		V(\bar{X}) &= \frac{\sigma^{2}}{n} \\
		\sigma^{2} &= V(\bar{X})n \\
		&= 4.055 \cdot 2 \\
		&= \boxed{8.11}
	\end{align*}
\end{center}

\textbf{12. (10 points) The life of a bread-making machine is a random variable with a mean of 7 years and a standard deviation of 1 year. A random sample of 36 machines is chosen. Find the probability that the...}

\qquad \textbf{a) ...average life of the sample is less than 7.45 years.}
\begin{center}
	Since $n$ is greater than 30, we can use the Central Limit Theorem:
	$$\text{mean} = \mu$$
	$$\text{variance} = \frac{\sigma^{2}}{n} = \frac{1^{2}}{36} = \frac{1}{36}$$
	$$\text{standard deviation} = \sqrt{\text{variance}} = \sqrt{\frac{1}{36}} = \frac{1}{6}$$
	\begin{align*}
		\phi \left(\frac{x - \mu}{\sigma}\right) &= \phi \left(\frac{7.45 - 7}{\frac{1}{6}}\right) \\
		&= \phi(2.70) \\
		&= \boxed{.9965}
	\end{align*}
\end{center}

\pagebreak

\qquad \textbf{b) ...average life of the sample is between 6.65 and 7.45 years.}
\begin{center}
	\begin{align*}
		\phi \left(\frac{7.45 - 7}{\frac{1}{6}}\right) - \phi \left(\frac{6.65 - 7}{\frac{1}{6}}\right) &= \phi(2.70) - \phi(-2.10) \\
		&= 0.9965 - .0179 \\
		&= \boxed{.9786}
	\end{align*}
\end{center}

\textbf{14. (5 points) The time until recharge for a battery in a laptop computer is normally distributed with mean of 260 minutes and standard deviation of 50 minutes. If an office has 4 laptop computers, find the probability that the total time the batteries lasted were between 868 and 1189 minutes.}
\begin{center}
	Even though $n$ is less than or equal to 30, we are told that we are allowed to use a normal distribution. Because there are 4 different laptops, we need to find the probability of the sample total being between 1189 and 868. This means the average laptop should last a quarter of these times:
	$$\frac{1189}{4} = 297.25$$
	$$\frac{868}{4} = 217$$
	\begin{align*}
		\phi \left(\frac{x_{2} - \mu}{\sigma}\right) - \phi \left(\frac{x_{1} - \mu}{\sigma}\right) &= \phi \left(\frac{297.25 - 260}{50}\right) - \phi \left(\frac{217 - 260}{50}\right) \\
		&= \phi(0.75) - \phi(-0.86) \\
		&= .7734 - .1949 \\
		&= \boxed{.5785}
	\end{align*}
\end{center}

\pagebreak

\textbf{16. (5 points) Components of type $A$ have heights that are independently distributed as a normal distribution with a mean 190 and a standard deviation of 10. Components of type $B$ have heights that are independently distributed as a normal distribution with a mean 150 and a standard deviation of 8. What is the probability that a stack of four components of type $A$ placed one on top of the other will be taller than a stack of five components of type $B$ placed one on top of the other?}
\begin{center}
	This problem is essentially asking for the probability that the average height of five $B$'s is less than the average height of four component $A$'s:
	$$4 \cdot (\text{mean height of component }A) = 760$$
	Dividing this number by 5 will give us the threshold that the average of the 5 $B$'s cannot go over:
	$$760 / 5 = 152$$
	Now we find the probability that the average height of five component $B$'s is less than or equal to this number:
	\begin{align*}
		\phi \left(\frac{x - \mu}{\sigma}\right) &= \phi \left(\frac{152 - 150}{8}\right) \\
		&= \phi(0.25) \\
		&= \boxed{.5987}
	\end{align*}
\end{center}

\textbf{18. (10 points) Let $X_{1}$, $X_{2}$, $X_{3}$, and $X_{4}$ be four independent, normal random variables with means 14, 12, 18, and 26 and standard deviations 4, 5, 2, and 3, respectively.}

\qquad \textbf{a) Find $P(46<X_{1}+X_{2}+X_{3}+X_{4}<88)$.}
\begin{center}
	$$\mu = 14 + 12 + 18 + 26 = 70$$
	$$\sigma^{2} = 4^{2} + 5^{2} + 2^{2} + 3^{2} = 54$$
	$$\sigma = \sqrt{54} = 7.35$$
	\begin{align*}
		\phi \left(\frac{88 - 70}{7.35}\right) - \phi \left(\frac{46 - 70}{7.35}\right) &= \phi(2.45) - \phi(-3.27) \\
		&= .9929 - .0005 \\
		&= \boxed{.9924}
	\end{align*}
\end{center}

\qquad \textbf{b) Find $P(22<3X_{1}-2X_{2}+4X_{3}-X_{4}<120)$.}
\begin{center}
	$$\mu = 3(14) - 2(12) + 4(18) - 1(26) = 64$$
	$$\sigma^{2} = 3(4^{2}) - 2(5^{2}) + 4(2^{2}) - 1(3^{2}) = 5$$
	$$\sigma = \sqrt{5} = 2.24$$
	\begin{align*}
		\phi \left(\frac{120 - 64}{2.24}\right) - \phi \left(\frac{22 - 64}{2.24}\right) &= \phi(25) - \phi(-18.75) \\
		&= 1 - 0 \\
		&= \boxed{1}
	\end{align*}
\end{center}
\end{document}